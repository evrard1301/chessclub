\begin{frame}{Présentation}

  \begin{center}
    \includegraphics[scale=0.5]{img/logo_ufc.png}
    \hspace{1em}
    \includegraphics[scale=0.3]{img/logo_ocr.png}
  \end{center}
  
  \begin{block}{Bérenger Ossété Gombé}
    \begin{itemize}
    \item Master 1 en informatique (2016-2017)
      \begin{itemize}
      \item Université de Franche-Comté (UFR-ST)
      \item Spécialité génie logiciel
      \end{itemize}
    \item Formation Python chez OpenClassRooms depuis janvier 2022
    \end{itemize}    
  \end{block}
\end{frame}

\begin{frame}{Le projet: développer une application python}

  \begin{block}{Objectifs}
    \begin{itemize}
    \item Produire du code robuste
    \item Utiliser la POO
    \item Structurer un projet python
    \end{itemize}
  \end{block}
\end{frame}

\begin{frame}{Contexte fictif: le club d'échecs}
  \begin{figure}
    \begin{center}
      \includegraphics[scale=0.2]{img/logo_club.png}    
    \end{center}    
    \caption{Logo du club d'échecs}    
  \end{figure}
  
  \begin{block}{Personnages}
    \begin{itemize}
    \item Nous incarnons un développeur indépendant
    \item Quelques membres du club d'échecs
      \begin{itemize}
      \item Élie, notre amie
      \item Édouard, l'oganisateur
      \item Charlie, l'assistant informatique
      \end{itemize}
    \end{itemize}
  \end{block}
\end{frame}

\begin{frame}{Contexte fictif: le club d'échecs}
  \begin{block}{Problématique}
    \begin{itemize}
    \item «\textit{Nous utilisons actuellement une \textbf{application
        en ligne} pour nous aider à gérer nos tournois d'échecs
      hebdomadaires. Malheureusement, cette application nous a déçus par
      le passé. Elle \textbf{tombe souvent en panne}, ce qui signifie
      que les matchs sont retardés.}» - \underline{Spécification technique}

    \item «\textit{Après que nous avons eu terminé le premier tour,
      \textbf{Internet a cessé de fonctionner} et le directeur du tournoi n'a
      pas pu entrer les scores sur le site des résultats}» - \underline{Témoignage d'Élie}
    \end{itemize}
  \end{block}
\end{frame}

\begin{frame}{Le besoin client}
  \begin{block}{Exigences fonctionnelles}
    \begin{itemize}
    \item Gestion des tournois
    \item Gestion des classements
    \item Génération de rapports pouvant être exportés\footnote{Il faut permettre cette future fonctionnalité.}
    \item Fonctionnement hors-ligne
    \item Interaction utilisateur \textit{via} une interface console
    \item Persistance \textit{via} une base de données
    \end{itemize}
  \end{block}
\end{frame}

\begin{frame}{Le besoin client}
  \begin{block}{Exigences non-fonctionnelles \footnote{Extraites de la spécification technique.}}
    \begin{itemize}
    \item \textbf{Maintenabilité}
      \begin{itemize}
      \item «\textit{Le code doit être \underline{aussi propre et maintenable que
        possible} pour éviter les bugs.}»
      \end{itemize}
    \item \textbf{Fiabilité}
      \begin{itemize}
      \item «\textit{Le code doit être aussi propre et maintenable que
        possible pour \underline{éviter les bugs}.}»
      \end{itemize}

    \item \textbf{Portabilité} (GNU/Linux, Windows, MacOS)
      \begin{itemize}
      \item «\textit{Le programme devrait fonctionner sous
        \underline{Windows, Mac ou Linux}}»
      \end{itemize}

    \item \textbf{Utilisabilité}
      \begin{itemize}
      \item «\textit{Tant que le programme affiche les résultats du
        tournoi \underline{proprement}, nous serons heureux !}»
      \end{itemize}      
    \end{itemize}
  \end{block}
\end{frame}
