\subsection{Agilité et tests}
\begin{frame}{Agilité}
  \begin{block}{Mise en place du backlog}
    \begin{itemize}
    \item Utilisation d'un tableau Kanban \textit{via} kanboard \footnote{\url{https://kanboard.org/}}
    \item Découpage des \textit{user stories} avec critères d'acceptations
    \end{itemize}
  \end{block}
\end{frame}

\begin{frame}{Tests}
  \begin{block}{Pourquoi tester ?}
    \begin{itemize}   
    \item La fiabilité est un enjeu du projet
      \begin{itemize}
      \item «\textit{Le code doit être aussi propre et maintenable que
        possible \textbf{pour éviter les bugs}.}»\footnote{D'après la
      spécification technique.}
      \end{itemize}
    \item L'algorithme du tournoi Suisse doit être fiable car au cœur
      du domaine métier
    \end{itemize}
  \end{block}
  
  \begin{block}{Outils de tests}
    \begin{itemize}
    \item Tests unitaires \textit{via} PyTest \footnote{\url{https://docs.pytest.org/en/7.1.x/}}
    \item Tests d'acceptations \textit{via} Behave
      (cucumber) \footnote{\url{https://behave.readthedocs.io/en/stable/}}
    \end{itemize}    
  \end{block}
\end{frame}

\subsection{Style et conventions}
\begin{frame}{Style et conventions}
  \begin{block}{PEP 8}
    \begin{itemize}
    \item Vérification de la conformité du programme avec PEP 8
      \textit{via}
      Flake8\footnote{\url{https://flake8.pycqa.org/en/latest/}}
    \end{itemize}
  \end{block}
  
  \begin{block}{Google et OpenStack}
    \begin{itemize}
    \item Google Python Style Guide \footnote{\url{https://google.github.io/styleguide/pyguide.html}}
    \item OpenStack Style
      Guidelines \footnote{\url{https://docs.openstack.org/hacking/latest/user/hacking.html}}
    \item Utilisation de l'extension
      hacking\footnote{\url{https://pypi.org/project/hacking/}} de
      Flake8
    \end{itemize}
  \end{block}
\end{frame}

\subsection{Bibliothèques Python}
\begin{frame}{Bibliothèques Python}
  \begin{block}{Bibliothèques}
    \begin{itemize}
    \item Style et conventions
      \begin{itemize}
      \item Flake8
      \item Flake8-html
      \item Hacking      
      \end{itemize}

    \item Tests
      \begin{itemize}
      \item PyTest
      \item Behave
      \end{itemize}

    \item Base de données
      \begin{itemize}
      \item TinyDB
      \end{itemize}

    \item Parseur XML
      \begin{itemize}
      \item BeautifulSoup
      \end{itemize}
    \end{itemize}
  \end{block}
\end{frame}
